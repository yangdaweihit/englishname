\documentclass[UTF8,a4paper]{article}
\usepackage{ctex}
\usepackage{amsmath}
\usepackage{marvosym}
\usepackage{cprotect}
\usepackage{tabularx}
\usepackage{graphicx}
\usepackage[top=2.5cm,bottom=2.5cm,marginparwidth=2cm,vmargin=3cm]{geometry}
\usepackage[unicode,colorlinks,linkcolor=blue,citecolor=red,bookmarksnumbered,bookmarks=true]{hyperref}
\usepackage{marginnote}
\newcommand\mycite[1]{\raisebox{1.0ex}{\zihao{6}\cite{#1}}}

\renewcommand\refname{\center 参考文献}

\renewcommand{\baselinestretch}{1.6}
\newcommand\note[1]{\marginpar{\zihao{6} \color{red} {#1}}}
\graphicspath{{figures/}}
\usepackage{syntonly}
%\syntaxonly
\bibliographystyle{unsrt}
\title{科研文献中的人名}
\author{杨大伟 \\ {\it yangdawei.hit@gmail.com}}

\date{\today}
%-----------------------------------------------------------------------------------------------
\begin{document}

\maketitle

\section{前言}
在科研文献中署名和引用人名的规则对我来说一直不是很清楚,于是今天写报告前决定花点时间看看规则。找来《芝加哥风格手册》\mycite{chicago}学习,算是有点清楚了,记下来备忘。文中括号内的数字表示规则在手册中对应的条款。如果你发现我的理解或某些细节有错误\Frowny,请写信教我改正错误\Smiley。

文献\cite{chicago}下载自\url{http://ishare.iask.sina.com.cn/f/6194855.html}。当然如果想买正版,可以访问\url{http://www.chicagomanualofstyle.org/books.html}。

\section{英美人姓名的组成}
英美人的姓名由三部分组成:名(first name\ 或\ given name),中间名(second name\ 或\  middle name),姓(last name\ 或\ surname\ 或\ family name)\mycite{name}\mycite{wiki}。比如美国现任总统奥巴马的全名:

\Pointinghand
\begin{equation*}
\underbrace{\bf Barack}_{\color{blue} given\ name} \overbrace{\bf Hussein}^{{\color{blue} second\ name}} \underbrace{\bf Obama}_{\color{blue} family\  name}
\end{equation*}
所以我们在学术交流中提到的某位学者,所称的实际上是他的姓。中国人是没有中间名的,故省略掉。对应英美的习惯,我们的名就是given name,姓就是family name。拿我的名字来说应该写为:

\Pointinghand
\begin{equation*}
\underbrace{\bf Dawei}_{\color{blue} given\ name} \overbrace{\bf Yang}^{{\color{blue} family\  name}}
\end{equation*}
在(18.76)中关于中国人的姓名,手册规定了两种方式:一种即是示例样式;另一种即姓一直在前面,名字(若两个字)中两字之间用短划线,如“Yang Da-wei”。可是手册并没有说明何种情况下用这两种样式。我个人倾向在标题下署名是采用后一种形式,而在文献中采用英美人的习惯。

\section{名字的缩写}
名字缩写即仅名字仅写首字母(initial),并以后面跟一个点表示缩写。(17.23)在三个名中,优先缩写的中间名(second name),其次是名(given name),姓(family name)一般是不缩写的,比如“Berack H. Obama”或“B. H. Obama”。我的名字缩写应该是“D. Yang”。注意,在缩写字母之间要留有一个空格。

当姓名列于著作封面或题目之下时,一般是按前面提到的顺序写姓名的。但如果把姓名写在参考文献列表中(仅是首位作者)或在文章中引用姓名时,需要将姓提到前面,并在后面跟一个逗号,如“Yang, D.”或“Obama, B. H.”。如果点和逗号也要省掉,那就写成“Yang D”或“Obama BH”。(17.24)我们称这种写法为“反写”(inverted name)。

还有一种情况---即姓全部大写,因此看到全部大写可以马上认出这是他(她)的姓。这是有些期刊采用的办法,因为很难记得清地球上那么多民族的姓氏表达习惯,就干脆把姓大写,方便识别。国外也有复姓,即两个名字合成的姓,表达复姓是两个名字之间加入短划线,如“Abbas Emami-Naeini”。

\section{参考文献中的合作者署名}
当参考文献中只有一位作者时,则作者名字不缩写。(17.26)

在参考文献中,两位以上作者的姓名连续写出,其中第一位作者姓名反写,其余作者正写。在姓名与姓名之间用逗号隔开,最后一名作者姓名前还要写上“and”。(17.27,17.28,17.29)比如我和奥巴马一同起草了一份中美关系声明,则我们的声明在参考文献中是看起来是这样的:

\Pointinghand {\color{blue}\ \ \ Yang, D.,\ and B. H. Obama.\ 2012.\ Good good study, day day up. 2012} 

如果合作者过多,一般认为列出前三位姓名,以后的作者省略掉,用“et al.”表示。(17.30)另外记一句,在参考文献中各元素之间用点(period)隔开,在笔记中则用逗号隔开。我们在维基中可以看到就是采用逗号隔开各元素的,但大多时候我们采用的是前一种方式。(16.15)

\section{正文中引用人名}
在正文中引用人名一般是使用姓,即family name。当有作者的姓相同时,才加上名或名的首字母缩写。(16.44)正文引用人名一般出现在文献引用中,文献引用使用的是所谓“作者--日期系统”(author-date system),即在一对括号中首先写出作者姓,然后是文献出版年,如果需要再给出引用页码,如(Yang 2012)。如果人名出现在正文中,则把人名的姓放在括号前面,如Yang(2012)。(16.4)

\begin{thebibliography}{99}
\bibitem{chicago} David Spencer. Chiago Manual of Style, 16th Edition. University Chicago Press, 2011.
\bibitem{name}\url{http://www.behindthename.com/glossary/view/english_names}
\bibitem{wiki}\url{http://en.wikipedia.org/wiki/English_name}
\end{thebibliography}

\end{document}

